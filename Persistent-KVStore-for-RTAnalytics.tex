% 
%% Replace 'XX' with your paper number (assigned when you register abstract)
%% Replace 'NN' with actual number of pages. 

\documentclass[10pt,preprint,numbers]{sigplanconf}
%========================
%  Packages
%========================
\usepackage{times}
\usepackage{datetime}
\usepackage{url}
\usepackage{hyperref}
\usepackage{color}

%========================
%  Macros
%========================

\newcommand{\code}[1]{\textsf{\fontsize{9}{11}\selectfont #1}}

\newcommand{\inred}[1]{{\color{red}{#1}}}
\newcommand{\remove}[1]{}
\newcommand{\Idit}[1]{\inred{[Idit: #1]}}

\newcommand{\sys}{PiWi}

\newcommand{\tuple}[1]{\ensuremath{\langle \mbox{#1} \rangle}}



\conferenceinfo{SOSP'17}{October 29--31, 2017, Shanghai, China}
\copyrightyear{2017} 

% These only appear when the 'preprint' option is specified.
% Enabling these will cause the first page of the document to fail the 
% format check on HotCRP :-(
\titlebanner{Under submission to SOSP 2017 - do not cite or distribute}
\preprintfooter{Draft of {\currenttime}, \today{}}

% No date in title area.
\date{}

% Paper number and no. of pages as author
\authorinfo{Paper \textbf{\#XX}}{NN pages}

% Actual document begins below.
\begin{document}

\title{\sys: Persistent Key-Value Storage for Real-Time Analytics} 
\maketitle

\begin{abstract}

TBD

\end{abstract}

\section{Introduction}

% Motivating examples
Real-time analytics is super important. KV-stores are essential building blocks for this.
\Idit{The paragraph below is copy-paste from KiWi, need to reword.}
KV-stores underlie web-scale data processing systems such as
Google's F1~\cite{Shute2013}, which powers its AdWords
%\footnote{\url{https://www.google.com/adwords/}}
business, and Yahoo's Flurry~\cite{flurry} --
%\footnote{\url{https://developer.yahoo.com/flurry/docs/analytics/}},
the technology behind Mobile Developer Analytics.
For example,
as of early 2016,  Flurry  reported systematically collecting data of $830\!,000$ mobile
apps~\cite{appmatrix}
%\footnote{\url{http://flurrymobile.tumblr.com/post/144245637325/}}  % appmatrix
running on 1.6 billion
user devices~\cite{phablet}.
%\footnote{\url{http://flurrymobile.tumblr.com/post/117769261810/}}. % the-phablet-revolution
Flurry streams  this data into a massive index, and provides
%application developers with tools that produce
a wealth of reports over the collected data.
Such
\emph{real-time analytics}  applications push KV-store scalability requirements to new levels and raise novel use cases.
Namely, they require
%  real-time performance for 
both
(1) low latency persistent ingestion of incoming data, and (2) high performance analytics of the resulting dataset.

% Goals
We emphasize three key goals for the service: First, for analytics, \emph{consistency} is important, namely, atomic scans. 
Second, \emph{persistence} is essential, namely, no updates should be lost due to crash, which inevitably do happen from time to time\cite{?}.
Finally, for real time ingestion, \emph{update performance} is a key consideration. 

% Why not LSM
LSM trees are really popular, but they induce high write amplification because they rewrite all data multiple times. In particular,
all files in a given layer undergo compaction, even `cold' ones that have not changes since the most recent update.
For persistence updates are additionally performed to a separte disk holding a \emph{write-ahead-log (WAL)}, which increases
the number of disk writes even further. LSM includes a memory component and a disk component, and all scans are served
by the disk component, assisted by a disk cache. 
We forgo the  LSM approach, and instead use a common data sturcture for the memory and disk components, 
encompassing both the WAL and the disk cache. 
Our approach reduces compaction by selectively compacting only
updated parts of the data structure, and not `cold' parts.
It further reduces write amplification and I/O bottlenecks by eliminating the need for a WAL.
Moreover, scans of `hot' key ranges may be served from the in-memory data structure without accessing disk.

% Results
We compare \sys\ to state-of-the-art key-value stores with awesome results. 

\section{Related Work}
\label{sec:related}

\section{\sys\ Design}
\label{sec:design}


\section{Evaulation}
\label{sec:eval}

To drive the benchmarks, we use the 
YCSB framework in C++\footnote{\url{https://github.com/basicthinker/YCSB-C}}  

We compare \sys\ to 
\begin{enumerate}
\item
LevelDB\footnote{\url{http://leveldb.org/}}
\item
RocksDB\footnote{\url{http://rocksdb.org/}}
\item
RocksDB with cLSM extensions\footnote{\url{https://github.com/guyg8/rocksdb}}
\end{enumerate}
All three systems are configured to provided persistence, namely, no data loss, by using their synchronous logging mode.
In addition, we deploy a single partition in each of the systems in order to provide consistent scans across the entire data store.

We do not compare with Tucana~\cite{tucana} as its code has not been available; moreover, it does not support the persistency and consistency guarantees that \sys\ provides. 
%https://www.usenix.org/conference/atc16/technical-sessions/presentation/papagiannis 
%cLSM https://subversion.assembla.com/svn/yahoo-lsm-ssds/
%http://dl.acm.org/citation.cfm?id=2741973 

\section{Conclusions}
\label{sec:conclusions}
\sys\ rocks.

\bibliographystyle{acm}
 \bibliography{ref}

\end{document}
