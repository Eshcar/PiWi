

%common LSM stores RocksDB, scylladb, HyperLevelDB, LevelDB, hbase, cassandra
The vast majority of industrial mainstream NoSQL KV-stores are  implemented as LSM trees~\cite{hbase, 
RocksDB, scylladb, Bigtable2008, cassandra2010}, building on the foundations set by O'Neil 
et al.~\cite{DBLP:journals/acta/ONeilCGO96, Muth1998}. 

Due to LSM's design popularity, much effort has been invested into working around its bottlenecks.
A variety of compaction strategies has been implemented in production systems~\cite{CallaghanCompaction, 
ScyllaCompaction} and research prototypes~\cite{triad, PebblesDB, vttrees, slmdb}. Other suggestions include storage
optimizations~\cite{WiscKey, PebblesDB, vttrees, slmdb}, boost of in-memory parallelism~\cite{scylladb, clsm2015}, and leveraging 
 workload redundancies to defer disk flushes~\cite{triad, accordion}. 

A number of systems focus on reducing write amplification.
PebblesDB~\cite{PebblesDB} introduces fragmented LSM trees in which level files are 
sliced into {\em guards\/} of increasing granularity and organized in a skiplist-like layout. This structure 
reduces write amplification. In contrast, \sys\/ eliminates the concept of levels altogether, 
and employs a flat storage layout instead. WiscKey~\cite{WiscKey} separates key and value storage 
in SSTables, also in order to reduce amplification. This optimization is orthogonal to \sys's concepts,
and could benefit our work as well. 
VT-Trees~\cite{vttrees} are LSM-trees that apply a stitching policy to avoid rewriting already sorted data. This improves performance and significantly reduces write amplification 
in some scenarios (e.g., time-series and filesystem metadata workloads). 
SLM-DB~\cite{slmdb} reduces write amplification by having only a single level of SSTable files on disk, supporting a selective compaction policy, and
exploiting persistent memory to eliminate the WAL. To expedite reads, it maintains a B$^+$-tree index that points to KV pairs on disk. Scans benefit from the spatial locality of keys in the B$^+$-tree leaves. Nevertheless, LSM-like temporal partitioning is used, and the actual values are retrieved from disk.

In-memory compaction has been implemented in HBase~\cite{accordion}, an LSM system.
The in-memory write store itself is organized as LSM tree, which helps eliminating redundancies 
in RAM to reduce disk flushes. However, being incremental to LSM design, 
this approach fails to address spatial locality. 

\sys's design resembles classic B-trees~\cite{Knuth:1998:ACP:280635}, which suffer from a write bottleneck in random updates to 
leaf blocks (similar to chunks). \sys\/ resolves this limitation through (1) transforming random I/O to sequential I/O at chunk level
(funks), (2) managing a write-through chunk cache in memory (munks), and (3) I/O reduction through munk and funk compactions. 

A $B^{\epsilon}$-tree ~\cite{Brodal:2003:LBE:644108.644201}is a B-tree variant that uses overflow write buffers in internal nodes. 
This design speeds up writes and reduces write amplification, however lookups are slowed down by having to search in unordered 
buffers. $B^{\epsilon}$-trees have been used in KV-stores (TokuDB~\cite{TokuDB}) and filesystems (BetrFS~\cite{BetrFS}).  
~\cref{sec:design} compares $B^{\epsilon}$-tree concepts to \sys. 

Tucana~\cite{tucana} is an in-memory $B^{\epsilon}$-tree index over a persistent log of KV-pairs. It applies multiple system
optimizations to speed up I/O (all orthogonal to our work): block-device storage access, memory-mapped files, and copy-on-write 
on internal nodes. However, Tucana provids neither strong scan semantics nor consistent recovery. 

In-memory KV-stores~\cite{ignite, redis, memcached, Srinivasan:2016:AAR:3007263.3007276} have originally emerged as fast volatile 
data storage, e.g., for web and application caches. Over time, many of them evolved to support durability,
%albeit as a second-class citizen in most cases. 
albeit they still require the complete data set to be memory resident.
%For example, Ignite~\cite{ignite} uses a B-tree index with in-place random updates. 
%These systems resemble \sys\/ in their memory-centric approach. 
We are unaware of the consistency guarantees or performance optimizations with respect  to disk-resident data in such systems. 

%Their persistent storage support is based on either B-tree~\cite{ignite} or LSM-tree design~\cite{redis}. 
%used for application data caching, but also as building blocks for distributed database with optional durability~\cite{ignite,redis}. These are not comparable with \sys\ as they are either not persistent~\cite{memcached}, do not support atomic scans~\cite{redis} or resemble  relational DBMS with a centralized WAL, , and in-place updates~\cite{ignite}  more than a KV LSM store.