

\subsection{API and guarantees}

\sys\ is a persistent key-value store supporting \emph{put, get}, and atomic \emph{range scan} (or scan) operations. 
Scans are atomic in the sense that all values returned by a single scan belong to a consistent snapshot reflecting
the state of the data store at a unique point in time.

\Idit{More on API?}

\sys\ ensures \emph{durability} of all updates by writing updates to disk synchronously as part of the \emph{put} operation.

\subsection{Organization and layout}

Similarly to BTrees \Idit{and other disk-friendly data structures?}, 
\sys\ organizes data in fixed-size \emph{chunks}, each holding a contiguous key range.

For persistence, all chunks are stored on-disk. Each chunk is associated with a \emph{file chunk}, or \emph{funk},
which is a persistent data structure consisting of three files holding the chunk's data -- a value store \emph{vstore}, 
a sorted key store \emph{kstore}, and a write buffer \emph{wbuf}. The vstore holds all the values associated with keys
in the chunk. When a funk is created, the kstore holds all the chunk's keys with pointers to corresponding values.
New keys are subsequently added to the unsorted wbuf.
The data store can be consistently recovered from the on-disk funks at any time. \Idit{Need to discuss recovery somewhere.}

A subset of the chunks is also cached in memory to allow fast access, each in a data structure called \emph{memory chunk (munk)}. 
\Idit{Explain munk structure.}

At run-time, \sys\ holds in memory chunk objects representing all funks in the data store. Each chunk holds 
a pointer to the appropriate funk, and, if applicable, also munk. In addition, the chunk holds a  synchronization data
structure called PPA, whose function will be explained below. 
Chunks are organized in a linked list, and are also indexed in-memory for fast access by key.

\Idit{Typical sizes: in-memory chunk object up to 1KB, chunk holding 10K to 100K keys, munk/funk size 1M to 100M depending 
on data size and number of keys, 1000s of chunks per map.}

\Idit{Discuss global version, PSA.}

\Idit{Add picture.}


  



%Since \sys\ is geared towards analytics workloads, which emphasize long range scans, we organize it as a sorted data structure.

