Many applications of key-value (KV-)storage exhibit high spatial locality
of access e.g., when data items with identical composite key prefixes are created or scanned together.
A prominent example at Verizon Media is Flurry Analytics which aggregate mobile application usage statistics typically 
identified by a composition of app id, device, time, location, event id, etc. The query API is app id-centric, hence this dimension is the key prefix.
This prevalent access pattern is underused by the ubiquitous LSM tree design underlying major KV-stores today.

We present \sys, a general-purpose persistent KV-store optimized for spatially-local workloads.
\sys\ forgoes the temporal data organization of LSM trees, and partitions data by key to exploit locality.
\sys's design also reduces write amplification, scales on multicore platforms and guarantees strong (atomic) semantics for updates, lookups, and scans. 
Our experiments with Flurry workload show that \sys\ consistently outperforms the state-of-the-art performance by 2.4x to 3.8x in ingestion rate and by 20\% to 90\% in scan speed.
Furthermore, traditional microbenchmarks show a lift in performance of up to 3.5x (1) in cases of high spatial locality; and (2) when the system has sufficient DRAM
to hold most of the active working set.

\remove{
Many applications of key-value (KV-)storage exhibit high \emph{spatial locality}
of access, for example, when data items with identical composite key prefixes are created or scanned together.  
This prevalent access pattern is underused by the ubiquitous LSM tree design underlying KV-stores today.

We present \sys, a general-purpose persistent KV-store optimized for spatially-local workloads. 
\sys\ forgoes the temporal data organization of LSM trees, and partitions data by key to exploit locality. 
%
\sys's design also reduces write amplification and expedites in-memory operation.
Our experiments show that \sys\ consistently outperforms the state-of-the-art performance 
(by up to 3.5x) (1) in cases of high spatial locality; and (2) when the system has sufficient DRAM 
to hold most of the active working set. 
In traditional YCSB workloads with larger working sets, % than the available  DRAM, \
\sys\ is on par with today's leading designs.
\sys\/ provides low write amplification, scales on multicore platforms and guarantees strong (atomic) semantics for updates, lookups, and scans. 
%Finally, it provides consistent crash recovery semantics, with near-instant recovery time. 
}